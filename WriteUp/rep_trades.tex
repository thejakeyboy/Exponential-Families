\documentclass[12pt]{article}

\usepackage{amsfonts}
\usepackage{latexsym}
\usepackage{amsmath}
\usepackage{amssymb}
\usepackage{amsthm}
\usepackage[oldstylenums]{kpfonts}
\usepackage[ruled]{algorithm2e}

\usepackage{hyperref}
\usepackage{enumerate}
\usepackage{fancyhdr}

\textwidth6.5in

\setlength{\topmargin}{0in} \addtolength{\topmargin}{-\headheight}
\addtolength{\topmargin}{-\headsep}
\textheight8.5in
\setlength{\oddsidemargin}{0in}

\oddsidemargin  0.0in \evensidemargin 0.0in \parindent0em

\pagestyle{fancy}\lhead{} \rhead{\bf \thepage}
\chead{{\Large\textsc{Exponential Family Prediction Markets}}} \lfoot{} \rfoot{} \cfoot{}
\newtheorem{thm}{Theorem}[section]
\newtheorem{prop}[thm]{Proposition}
\newtheorem{lem}[thm]{Lemma}
\newtheorem{con}[thm]{Conjecture}
\newtheorem{cor}[thm]{Corollary}
\newtheorem{ex}[thm]{Example}
\newtheorem{qn}[thm]{Question}
\newcommand{\defn}{\bigbreak\noindent{\bf Definition. }}
\newcommand{\undefn}{\bigbreak}
\newcommand{\rmk}{\bigbreak\noindent{\bf Remark. }}
\newcommand{\unrmk}{\bigbreak}
%\newcommand{\proof}{\noindent{\bf Proof. }}
\newcommand{\unproof}{\hfill$\Box$\bigbreak}
\newcommand{\defeq}{\stackrel{\mathrm{def}}{=}}
\newcommand{\holds}[1]{\stackrel{?}{#1}}
\newcommand{\conv}{\mathrm{conv}}
\newcommand{\ie} {{\it i.e. }}
\newcommand{\eg} {{\it e.g. }}
\newcommand{\etal} {{\it et al. }}
\newcommand{\E}{\mathbb{E}}
\newcommand{\hmu}{\hat{\mu}}


\begin{document}

\section{Reinterpreting repeated trades in a market}
Let us suppose that traders in this market have exponential utility \[
U(w) = -\frac{1}{a} \exp(-aw).
\]
%
Here $a$ is the coefficient of risk aversion (higher means more risk averse, and the utility function is more concave).

Let $\delta_{1}^{*}$ be the optimal vector of shares the agent decides to trade on first entering the market with exponential family belief parametrized by natural parameter $\hat{\theta}$. Thus, his belief distribution is given by the pdf $$\exp\{\hat{\theta} \phi(x) - T(\hat{\theta})\}]$$ where $T(\hat{\theta}) =\int_{\mathcal{X}}\exp\{\hat{\theta} \phi(x)\}\ dx$ is the log partition function and $\phi(x)$ are the sufficient statistics.
On a subsequent entry into this market when the market state is $\theta'$, his optimal purchase $\delta_{2}^{*}$ is given by the solution of
$$\arg\max_{\delta_{2}} \E_{x\sim \hat{\theta}} U[\mathrm{payoff}]$$
Here, the payoff given eventual outcome $x$ is  $[\delta_{1}^{*}+\delta_{2}^{*} ]\phi(x) - C(\delta_{1}^{*} + \theta) + C(\theta)- C(\delta_{2}^{*} + \theta') + C(\theta')$. Here $\theta' = \theta+\delta_{1}^{*}+\delta'$ where $\delta'$ captures the movement of the share vector by other traders in the market. 
The expected utility for this payoff is as follows.
%%
%\[ U( \delta x - C(\delta + \theta) + C(\theta) ) = -\frac{1}{a} \exp(-a \delta x + a C(\delta + \theta) - a C(\theta)).
%\]
%%
\begin{eqnarray*}
&   & \arg\max_{\delta_{2}}\E\left[ (\delta_{1}^{*}+\delta_{2} )\phi(x) - C(\delta_{1}^{*} + \theta) + C(\theta)- C(\delta_{2} + \theta') + C(\theta') \right] \\
& = & \arg\max_{\delta_{2}}\int_{\mathcal{X}} -\frac{1}{a} [\exp(-a (\delta_{1}^{*}+\delta_{2} )\phi(x) \\
&  & + a C(\delta_{1}^{*} + \theta) -a C(\theta)+a C(\delta_{2} + \theta') -a C(\theta')) \exp\{\hat{\theta} \phi(x) - T(\hat{\theta})\}]\, dx \\
& = & \arg\max_{\delta_{2}}\int_{\mathcal{X}} -\frac{1}{a} [\exp\{-a \delta_{2} \phi(x)+a C(\delta_{2} + \theta') -a C(\theta') \\
&  & + a C(\delta_{1}^{*} + \theta) -a C(\theta)\} \exp\{(\hat{\theta}-a \delta_{1}^{*}) \phi(x) - T(\hat{\theta})\}]\, dx \\
& = & \arg\max_{\delta_{2}}\int_{\mathcal{X}} U[N(\delta_{2},\theta')] \exp\{(\hat{\theta}-a \delta_{1}^{*}) \phi(x) - T(\hat{\theta})+ a C(\delta_{1}^{*} + \theta) -a C(\theta)\}]\, dx 
\end{eqnarray*}

Here the first equality follows from the fact that we are taking expectation over the traders belief parameter $\hat{\theta}$ and the second equality follows simply from rearranging the factors of $\phi(x)$. And lastly we have written $-\frac{1}{a} [\exp\{-a \delta_{2} \phi(x)+a C(\delta_{2} + \theta') -a C(\theta')]$ as $U[N(\delta_{2},\theta')]$ where $N(\delta_{2},\theta')$ is the net payoff when $\delta_{2}$ shares are purchased when the current market state is $\theta'$. 

This is equivalent to maximizing expected utility of $N(\delta_{2},\theta')$, where expectation is taken with respect to an exponential family distribution over $\mathcal{X}$ parametrized by $\hat{\theta}-a \delta_{1}^{*}$, so long as 
$$\int_{\mathcal{X}} \exp\{(\hat{\theta}-a \delta_{1}^{*}) \phi(x) - T(\hat{\theta})+ a C(\delta_{1}^{*} + \theta) -a C(\theta)\}]\, dx = c\times \int_{\mathcal{X}} \exp\{(\hat{\theta}-a \delta_{1}^{*}) \phi(x)-T(\hat{\theta}-a \delta_{1}^{*})\} \, dx$$

%$$\int_{\mathcal{X}} \exp\{(\hat{\theta}-a \delta_{1}^{*}) \phi(x)-\psi(\hat{\theta}-a \delta_{1}^{*})\} \, dx =1$$
%That is
%$$\psi(\hat{\theta}-a \delta_{1}^{*})= \int_{\mathcal{X}} \exp\{(\hat{\theta}-a \delta_{1}^{*}) \phi(x)\}\ dx$$
In other words we want the following statement to hold.
\begin{eqnarray*}
\exists c: c\times\exp\{- T(\hat{\theta})+ a C(\delta_{1}^{*} + \theta) -a C(\theta)\}&=&
T(\hat{\theta}-a \delta_{1}^{*})\\
&=& \int_{\mathcal{X}} \exp\{(\hat{\theta}-a \delta_{1}^{*}) \phi(x)\}\ dx
\end{eqnarray*}

where $c$ is a constant with respect to $\delta_{2}$ and $x$ (but could potentially depend on $\hat{\theta}$, $a$ and $\delta_{1}^{*}$). This is true because the right hand side of the equation is a function of only $\hat{\theta}$, $a$ and $\delta_{1}^{*}$. Let $\Theta\defeq\hat{\theta}-a \delta_{1}^{*}  $ be the effective belief. Thus we have that the trader chooses his share vector as follows.
\begin{eqnarray*}
&&  \arg\max_{\delta_{2}}\int_{\mathcal{X}} U[N(\delta_{2},\theta')] \exp\{(\hat{\theta}-a \delta_{1}^{*}) \phi(x) - T(\hat{\theta})+ a C(\delta_{1}^{*} + \theta) -a C(\theta)\}\, dx \\
& = &\arg\max_{\delta_{2}}\int_{\mathcal{X}} U[N(\delta_{2},\theta')] \exp\{(\hat{\theta}-a \delta_{1}^{*}) \phi(x) - T(\hat{\theta}-a \delta_{1}^{*})\}]\, dx\\
& = & \arg\max_{\delta_{2}} \int_{\mathcal{X}} -\frac{1}{a} \exp\{-a \delta_{2} \phi(x)+a C(\delta_{2} + \theta') -a C(\theta')\} \exp\{\Theta\cdot \phi(x) - T(\Theta)\}\, dx\\
& = & \arg\max_{\delta_{2}} U\left[- C(\delta_{2} + \theta') + C(\theta') - \frac{1}{a} T(\Theta - a\delta_{2}) + \frac{1}{a} T(\Theta)\right]\\
\end{eqnarray*}
This follows from the argument detailed in in exp-util and simple substitution. Further,  the maximizer is $\delta_{2}^* = (\Theta - \theta') / (1+a)$, which moves the share vector to $\theta' + \delta_{2}^* = \frac{1}{1+a} \Theta + \frac{a}{1+a} \theta'$ which is a convex combination of the effective belief and the current market state.

In other words, an exponential utility maximizing trader who has belief $\hat{\theta}$ with prior exposure $\delta$ in a market  will behave identically as an exponential utility maximizing trader with belief $\hat{\theta}-a \delta$ and no prior exposure in the market. Here $a$ is the utility parameter. This means that financial exposure can be equivalently understood as changing the privately held beliefs.

Note that as the trader becomes more risk averse, the  belief is updated more aggressively. As $a \rightarrow 0$, the trader becomes more risk neutral and his effective belief is closer to his true belief. 

\subsection{Equilibrium in a market with multiple traders}
For each trader $i$, 
\begin{equation}\label{eq:eqbm}
\beta_{i}-a_{i}\delta_{i}=\theta+\sum_{j=1}^{n}\delta_{j}
\end{equation}�
Rewriting, we have for each trader $i$, 
$$\frac{\beta_{i}}{a_{i}}-\delta_{i}=\frac{1}{a_{i}}(\theta+\sum_{j=1}^{n}\delta_{j})$$
Thus,
$$\sum_{i=1}^{n}\left(\frac{\beta_{i}}{a_{i}}\right)-\sum_{i=1}^{n}\delta_{i}=\frac{1}{\sum_{i=1}^{n}a_{i}}\left(\theta+\sum_{j=1}^{n}\delta_{j}\right)$$
And
$$\sum_{j=1}^{n}\delta_{j}=
\frac{\sum_{i=1}^{n}a_{i}\sum_{i=1}^{n}\left(\frac{\beta_{i}}{a_{i}}\right)-\theta}{1+\sum_{i=1}^{n}a_{i}}$$
Substituting in Equation \ref{eq:eqbm} we have the following expression for the final market state.
$$\frac{\sum_{i=1}^{n}a_{i}}{1+\sum_{i=1}^{n}a_{i}}\left[\sum_{i=1}^{n}\left(a_{i}^{-1}\beta_{i}\right)+\theta\right]$$
\end{document}